\documentclass[a4paper, 12pt, twocolumn]{article}

\usepackage[spanish]{babel}
\usepackage{amsmath, amssymb, graphicx, float, hyperref}
\usepackage{caption, subcaption}
\usepackage{geometry}

\geometry{margin=1in}

\title{Resolución Numérica de la Ecuación de Schrödinger Dependiente del Tiempo}

\author{
	César de la Rosa Sobrino \\
	Universidad Nacional de Educación a Distancia (UNED) \\
	Física Cuántica I
}

\date{\today}

\begin{document}

\maketitle

\begin{abstract}
	Hola
\end{abstract}

\twocolumn

\section{Introducción}
La ecuación de Schrödinger dependiente del tiempo es fundamental en la mecánica cuántica, ya que describe cómo el estado cuántico de un sistema evoluciona a lo largo del tiempo. Resolver esta ecuación de forma numérica es crucial para simular sistemas complejos donde las soluciones analíticas no son posibles.

HOLA QUE TAL quiero entender un poquito como va esto
\section{Ecuaciones en Derivadas Parciales}
\input{ecuaciones_edp.tex}

\section{Propagación de una Onda Sinusoidal}
\input{onda_sinusoidal.tex}

\section{Ecuación de Schrödinger Dependiente del Tiempo}
\input{schrodinger_tiempo.tex}

\section{Discretización de la Ecuación de Schrödinger}
\input{diferencias_finitas.tex}

\section{Evolución Temporal de una Partícula Libre}
\input{particula_libre.tex}

\section{Evolución Temporal de una Partícula Viajera}
\input{particula_viajera.tex}

\section{Funciones del Oscilador Armónico}
\input{oscilador_armonico.tex}

\section{Sistema de Elección Propia}
\input{sistema_elegido.tex}

\section{Conclusiones}
\input{conclusiones.tex}

\bibliographystyle{plain}
\bibliography{referencias}

\appendix
\section{Código Fuente}
\input{codigo_fuente.tex}

\end{document}
